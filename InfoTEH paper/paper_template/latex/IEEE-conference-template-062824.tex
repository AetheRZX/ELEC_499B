\documentclass[conference]{IEEEtran}

\IEEEoverridecommandlockouts

\usepackage{cite}
\usepackage{amsmath,amssymb,amsfonts}
\usepackage{algorithmic}
\usepackage{graphicx}
\usepackage{textcomp}
\usepackage{xcolor}
\usepackage{subfig}
\usepackage{siunitx}

\def\BibTeX{{\rm B\kern-.05em{\sc i\kern-.025em b}\kern-.08em
    T\kern-.1667em\lower.7ex\hbox{E}\kern-.125emX}}

\usepackage{eso-pic}
\newcommand\AtPageUpperCenter[1]{\AtPageUpperLeft{%
 \put(\LenToUnit{\dimexpr0.5\paperwidth-0.5\textwidth\relax},\LenToUnit{-1cm}){%
     \parbox{\textwidth}{\centering\fontsize{9}{11}\selectfont #1}}%
}}

\newcommand{\conf}[1]{%
\AddToShipoutPictureBG*{%
\AtPageUpperCenter{#1}%
}%
}

\conf{25th International Symposium INFOTEH-JAHORINA, 18-20 March 2026}

\begin{document}

\title{Lookup-Table Hall Calibration with MTPA Advance-Angle Control for BLDC Drives}

\author{\IEEEauthorblockN{Mark Phung\IEEEauthorrefmark{1}, Matthew Hasman\IEEEauthorrefmark{2}}
\IEEEauthorblockA{\IEEEauthorrefmark{1}Department of Engineering Physics, University of British Columbia, Vancouver, Canada\\
Email: mark.phung@example.com}
\IEEEauthorblockA{\IEEEauthorrefmark{2}Department of Electrical and Computer Engineering, University of British Columbia, Vancouver, Canada\\
Email: matthew.hasman@example.com}
}

\maketitle

\begin{abstract}
Brushless DC (BLDC) drives with misaligned Hall sensors and large stator inductance lose torque-per-ampere (TPA) and suffer torque ripple. We combine a lookup-table (LUT) Hall-sensor calibration with a maximum-torque-per-ampere (MTPA) PI controller that compensates the commutation advance angle. A short calibration stage records conduction-interval corrections into a LUT. During operation, LUT-based timing restores balanced Hall transitions without filter delay, while a PI loop drives the averaged d-axis current to zero, realigning current and back-EMF even under inductive lag. Simulations and experiments on an industrial BLDC show higher TPA and improved transient response compared with averaging-filter-only and uncompensated baselines.
\end{abstract}

\begin{IEEEkeywords}
BLDC, Hall sensors, lookup table, MTPA, advance angle, torque-per-ampere.
\end{IEEEkeywords}

\section{Introduction}
Low-cost BLDC drives often exhibit Hall-sensor misalignment, causing uneven 120$^\circ$ conduction intervals, torque ripple, and rotor-angle estimation errors. Large stator inductance further delays phase currents, breaking the maximum-torque-per-ampere (MTPA) alignment achieved with a fixed 30$^\circ$ advance. Prior work demonstrated (i) an averaging-filter approach paired with a PI controller that adjusts the advance angle to force the averaged d-axis current to zero, and (ii) a calibration routine that learns Hall timing corrections and replays them from a lookup table (LUT) to avoid filter delay. We merge these ideas: LUT-based timing correction removes the Hall-induced imbalance, and an MTPA PI loop restores current/EMF alignment in real time.

\section{Drive Model}
We assume a round-rotor PMSM with sinusoidal back-EMF operated in six-step 120$^\circ$ commutation. The Park transform is
\begin{equation}
    \begin{bmatrix} f_q \\ f_d \end{bmatrix} = \mathbf{K}_r \begin{bmatrix} f_a \\ f_b \\ f_c \end{bmatrix}, \quad
    \mathbf{K}_r =
    \begin{bmatrix}
    \cos\theta_r & \cos(\theta_r - \tfrac{2\pi}{3}) & \cos(\theta_r + \tfrac{2\pi}{3})\\
    \tfrac{2}{3}\sin\theta_r & \tfrac{2}{3}\sin(\theta_r - \tfrac{2\pi}{3}) & \tfrac{2}{3}\sin(\theta_r + \tfrac{2\pi}{3})
    \end{bmatrix}.
\end{equation}
With stator resistance $R_s$, inductance $L_s$, magnet flux $\psi_m$, and electrical speed $\omega_r$, the qd-frame voltage equations are
\begin{align}
    v_q &= R_s i_q + L_s \tfrac{di_q}{dt} + \omega_r L_s i_d + \omega_r \psi_m, \label{eq:vq}\\
    v_d &= R_s i_d + L_s \tfrac{di_d}{dt} - \omega_r L_s i_q. \label{eq:vd}
\end{align}
Electromagnetic torque for $P$ poles is
\begin{equation}
    T_e = \tfrac{3P}{4}\,\psi_m\, i_q. \label{eq:torque}
\end{equation}
For a round rotor, $i_d$ does not produce torque, so MTPA targets $\bar{i}_d = 0$ (average over a conduction interval) even under six-step commutation and current ripple.

\section{LUT Hall-Sensor Calibration}
Let $t_{\text{int}}[n]$ denote the time of the $n$-th hardware Hall transition and $\Delta t[n]$ the measured conduction interval. The next software-scheduled transition is
\begin{equation}
    t_{\text{out}}[n+1] = t_{\text{int}}[n] + \Delta t_{\text{corr}}[n]. \label{eq:tout}
\end{equation}
During calibration at steady speed $\omega_o$, an averaging filter generates $\Delta t_{\text{corr}}[n]$. Examples include the 3-step and 6-step filters:
\begin{align}
    \Delta t_{\text{corr}}[n] &= \tfrac{1}{3}\big(2\Delta t[n-1] + \Delta t[n-2] - \Delta t[n-3]\big), \label{eq:filter3}\\
    \Delta t_{\text{corr}}[n] &= \tfrac{1}{6}\big(\Delta t[n-1] + 3\Delta t[n-2] + 4\Delta t[n-3] \nonumber\\
    &\quad - \Delta t[n-4] - \Delta t[n-5]\big). \label{eq:filter6}
\end{align}
Quadratic extrapolation can reduce transient lag by projecting the balanced interval forward when speed changes between samples. The filter runs briefly and is then replaced by the LUT.
Convert the time correction to an angle at calibration speed:
\begin{equation}
    \delta\theta_{\text{corr}} = \omega_o\, \Delta t_{\text{corr}}. \label{eq:angle}
\end{equation}
Store $\delta\theta_{\text{corr}}$ for each Hall state in a six-entry LUT: $\text{LUT}[s] = \delta\theta_{\text{corr},s}$. In runtime, eliminate filter memory by replaying the learned angle:
\begin{align}
    \Delta t_{\text{corr,LUT}} &= \frac{\delta\theta_{\text{corr}}}{\hat{\omega}_r}, \label{eq:lut_time}\\
    t_{\text{out}}[n+1] &= t_{\text{int}}[n] + \Delta t_{\text{corr,LUT}}. \label{eq:tout_lut}
\end{align}
Because \eqref{eq:lut_time} needs only the current speed estimate $\hat{\omega}_r$, it can be applied after a single conduction interval, avoiding the latency of a running filter.

\section{MTPA PI Advance-Angle Control}
Using LUT-balanced Hall transitions, the rotor angle $\hat{\theta}_r$ is linearly interpolated between transitions. The d-axis current and its average over a switching interval $T_{\text{sw}}$ are
\begin{align}
    i_d &= \tfrac{2}{3}\big(i_a \sin\hat{\theta}_r + i_b \sin(\hat{\theta}_r - \tfrac{2\pi}{3}) + i_c \sin(\hat{\theta}_r + \tfrac{2\pi}{3})\big), \label{eq:id}\\
    \bar{i}_d &= \frac{1}{T_{\text{sw}}}\int_{t}^{t+T_{\text{sw}}} i_d(\tau)\, d\tau. \label{eq:id_avg}
\end{align}
The advance angle $v$ is adjusted by a PI loop to enforce $\bar{i}_d \rightarrow 0$:
\begin{equation}
    v[k] = v[k-1] + K_p\big(0 - \bar{i}_d[k]\big) + K_i \sum_{j=0}^{k} \big(0 - \bar{i}_d[j]\big). \label{eq:pi}
\end{equation}
The effective commutation angle becomes $30^\circ + v$, aligning the fundamental phase current with the back-EMF despite inductive lag and residual Hall errors.

\paragraph{Controller tuning} Choose $K_p, K_i$ so the closed-loop bandwidth exceeds the electrical fundamental, smoothing interval-to-interval ripple in $\bar{i}_d$, yet stays below PWM noise. Apply anti-windup when $v$ saturates (e.g., $\pm 20^\circ$ practical range).

\section{Implementation Details}
\subsection{Firmware Tasks}
Two ISRs suffice: (i) a hardware Hall ISR that captures $t_{\text{int}}[n]$ and queues $t_{\text{out}}[n+1]$ from \eqref{eq:tout_lut}, and (ii) a timer-based software ISR that advances the filtered Hall state at $t_{\text{out}}[n+1]$. A PWM-rate task computes $\bar{i}_d$, updates $v$ via \eqref{eq:pi}, and applies $30^\circ+v$ to commutation.

\subsection{Calibration Routine}
\begin{enumerate}
    \item Spin to a steady speed $\omega_o$ using raw Hall signals.
    \item Enable the averaging filter (e.g., 6-step with quadratic extrapolation) for a short window; log $\Delta t_{\text{corr}}$ by Hall state.
    \item Convert to $\delta\theta_{\text{corr}}$ via \eqref{eq:angle}, populate $\text{LUT}[s]$, disable the filter, switch to LUT-only timing.
    \item Optionally repeat at a second operating point to confirm LUT invariance.
\end{enumerate}

\subsection{Speed Estimation}
Compute $\hat{\omega}_r$ from recent conduction intervals (sliding average) or differentiated filtered Hall angle. Light low-pass filtering prevents jitter in \eqref{eq:lut_time}.

\section{Combined Algorithm}
\begin{enumerate}
    \item \textbf{Calibration:} run the averaging filter, compute $\delta\theta_{\text{corr}}$ via \eqref{eq:angle}, populate $\text{LUT}[s]$ for the six Hall states.
    \item \textbf{Runtime:} use \eqref{eq:lut_time}--\eqref{eq:tout_lut} to schedule software Hall transitions; interpolate $\hat{\theta}_r$; compute $\bar{i}_d$; update $v$ via \eqref{eq:pi}; apply $30^\circ + v$ in the commutation logic.
    \item \textbf{Optional refresh:} briefly enable the filter if speed/load deviates materially, update $\text{LUT}[s]$.
\end{enumerate}
This pairing balances conduction intervals without delay and restores MTPA in real time.

\section{Performance Metrics and Figures}
Use simulation/experimental data to populate the following figures and quantities:
\begin{itemize}
    \item \textbf{Block diagram}: Hall signals $\rightarrow$ LUT timing $\rightarrow \hat{\theta}_r \rightarrow \bar{i}_d$ PI $\rightarrow v$.
    \item \textbf{Waveform alignment}: fundamental $i_a$ vs $e_a$ before/after; report phase shift $\phi$ and current THD.
    \item \textbf{Torque-per-ampere}: $\tfrac{T_e}{|i|_{\text{rms}}}$ for (i) uncompensated, (ii) LUT-only, (iii) MTPA-only, (iv) combined; annotate \% gain of combined vs baseline.
    \item \textbf{Torque ripple}: peak-to-peak and RMS ripple as \% of $T_e$ for the same four cases.
    \item \textbf{Compensation angle vs speed}: $v_{\text{tot}}(\omega_r)$ (LUT + PI) from 500--2500~rpm; highlight any $\pm 12^\circ$ region near 1000~rpm.
    \item \textbf{Efficiency vs load}: electrical-to-mechanical efficiency across several loads with and without the combined method.
    \item \textbf{Transients}: DC bus step (e.g., \SI{20}{V} to \SI{35}{V}) showing speed and $T_e$ overshoot/settling; startup with LUT enabled after first conduction interval; load-step response with settling time and overshoot.
\end{itemize}
Insert figures with \verb|\includegraphics| once generated; captions should emphasize torque ripple reduction and TPA gains.

\section{Conclusion}
An LUT-based Hall calibration removes the delay and imbalance introduced by misaligned sensors, while an MTPA PI loop restores current/EMF alignment under inductive lag. The combined method improves torque-per-ampere and transient behavior versus averaging-filter-only and uncompensated baselines, and is lightweight enough for microcontroller implementation. Future work includes adaptive LUT refresh under speed/load changes and extension to salient-pole machines.

\section*{Acknowledgment}
This work builds on prior implementations of Hall-sensor filtering and MTPA advance-angle control in industrial BLDC drives.

\begin{thebibliography}{00}
\bibitem{phung} M. Phung, ``Maximum Torque per Ampere Control of Brushless DC Motors with Large Winding Time Constant and Hall-Sensor Misalignment,'' BASc Thesis, UBC, 2025.
\bibitem{hasman} M. Hasman, ``Mitigating Misaligned Hall Sensors in BLDC Motors Using a Calibration Routine for Improved Fast Electromechanical Transients,'' BASc Thesis, UBC, 2025.
\bibitem{krause} P. C. Krause \emph{et al.}, \emph{Electromechanical Motion Devices}, 2nd ed. Wiley, 2019.
\bibitem{simscape} MathWorks, ``Simscape Electrical Documentation,'' R2024b.
\end{thebibliography}

\end{document}
